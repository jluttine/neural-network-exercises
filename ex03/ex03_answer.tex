\documentclass[10pt]{article}
\usepackage[english]{babel}
\usepackage[latin1]{inputenc}
\usepackage{subfigure}
\usepackage{epsfig}
\usepackage{amsmath,amssymb}
\parindent 0mm
\textwidth 16cm
\textheight 23cm
\oddsidemargin 0cm
\evensidemargin 0cm
\topmargin -10mm
%\newcommand{\vect}[1]{{\bf{#1}}}
\newcommand{\svect}[1]{\boldsymbol{#1}}
%\newcommand{\matr}[1]{\boldsymbol{#1}}
%\newcommand{\m}[1]{\boldsymbol{#1}}
\input macros.tex


\begin{document}
\pagestyle{empty}
\begin{Large}
\begin{bf} 
T-61.5130 Machine Learning and Neural Networks\\ 
\end{bf}
\end{Large}
Karhunen, Hao\\  
\\
\begin{large}
\begin{bf}
Exercise 3,  11.11.2011\\Model answer
\end{bf}
\end{large}
\begin{enumerate}

\item Consider a wide-sense-stationary
  first-order discrete-time Markov process given by the stochastic
  difference equation
\[
{\bf x}(n) = \alpha {\bf x}(n-1) + {\bf v}(k)
\]
where $\alpha = 0.9; {\bf x}, {\bf v} \in \mathbb{R}^{3\times 1}$,
and ${\bf v}$ is zero-mean Gaussian white noise with a covariance
matrix given as
\[ \textbf{C}_v = \left[ \begin{array}{ccc}
5 & 0 & 0 \\
0 & 3 & 0 \\
0 & 0 & 0.1 \end{array} \right]\]

a) Calculate the covariance matrix ${\bf C}_x = \text{E}\left[ {\bf
    xx}^T \right]$\\
b) Calculate the theoretical eigenvalues and eigenvectors of ${\bf C}_x$

---------------------------------------------------------------------------------------------

a) The covariance matrix for vector $\vect{x}$ is
\[
\begin{split}
\matr{C}_x &=
E[\vect{x} \vect{x}^T]=E[[\alpha\vect{x}(n-1)+\vect{v}(n)][\alpha\vect{x}(n-1)+\vect{v}(n)]^T]\\
&=\alpha^2 E[\vect{x}(n-1)\vect{x}(n-1)^T]+2 \underbrace{E[\alpha \vect{x}(n-1)\vect{v}(n)^T]}_{=0}+ E[\vect{v}(n)\vect{v}(n)^T]\\
%&=\alpha^2 E[\vect{x}(n-1)\vect{x}(n-1)^T]+2E[ \alpha \vect{x}(n-1)\vect{v}(n)^T]\\
&=\alpha^2 \matr{C}_x + \matr{C}_v
\end{split}
\]
which yields
\[
\matr{C}_x = \frac{1}{1-\alpha^2}\matr{C}_v=
\left[\begin{array}{ccc}
26.32 & 0     & 0 \\
0     & 15.79 & 0 \\
0     & 0     & 0.53 
\end{array}\right]
\]


b) Since the covariance matrix $\matr{C}_x$ is diagonal its
eigenvalues and eigenvectors are
\[
\begin{cases}
\lambda_1 &= 26.32 \\
\lambda_2 &= 15.79 \\
\lambda_3 &= 0.53
\end{cases}
\]

and eigenvectors

\[
\begin{cases}
\vect{e}_1 &= [1 \hspace{2mm} 0 \hspace{2mm} 0]^T \\
\vect{e}_2 &= [0 \hspace{2mm} 1 \hspace{2mm} 0]^T \\
\vect{e}_3 &= [0 \hspace{2mm} 0 \hspace{2mm} 1]^T
\end{cases}
\]

\vspace{2mm}

\vspace{2cm}
\item  a) Modify the standard Gram-Schmidt orthogonalization (GSO) method for
performing whitening. b) What can you say about the uniqueness of whitening in view of the
developed algorithm?

---------------------------------------------------------------------------------------------

a) Assume that we have m linearly independent $n$-dimensional ($n\geq
m$) vectors $\vect{a}_1 \vect{a}_2, ..., \vect{a}_m$, The standard Gram-Schmidt (GSO)
orhogonalization methods can be used to contruct a set $\vect{w}_1, ..., \vect{w}_m$
of $m$ vectors that spands the same subspace as the original
vectors. This means that each $\vect{w}_i$ is some linear combination of the
$\vect{a}_j$

The standard GSO method is as follows:

\[
\begin{split}
\tilde{\vect{w}}_1 &= \vect{a}_1 \\
\tilde{\vect{w}}_j &= \vect{a}_j - \sum_{i=1}^{j-1} (\vect{w}_i^T
\vect{a}_j)\vect{w}_i, \hspace{4mm} j=2,...,m \\
\vect{w}_j &= \tilde{\vect{w}}_j / \| \tilde{\vect{w}}_j \| \hspace{4mm} j=1,2,...,m
\end{split}
\]

It is easy to see that the vectors $\textbf{w}_1, ..., \textbf{w}_m$ are mutually
orthonormal: 

\[\vect{w}_i^T\vect{w}_j = \delta_{ij} = \begin{cases}1, &i=j\\0 &i\neq
j\end{cases}
\]

In matrix form

\[
\vect{W}^T\vect{W} = \matr{I}, \hspace{6mm}\vect{W} = [\vect{w}_1, \vect{w}_2, ..., \vect{w}_m]
\]

In whitening, the original data vectors $\vect{x}$ are transformed so
that the whitened vectors $\textbf{y}=\vect{W}^T\vect{x}$ satisfy the condition
$E\lbrace \textbf{y}\textbf{y}^T \rbrace=\matr{I}$.

This yields
$E\lbrace \vect{W}^T\vect{x}\vect{x}^T\vect{W}\rbrace=\vect{W}^T \textbf{C}_{xx} \vect{W}=\matr{I}$
where $\matr{C}_{xx}$ is the correlation matrix of $\vect{x}$
(assuming that $\vect{x}$ has zero mean).

For individual elements of the matrix $E\lbrace \vect{y}\vect{y}^T\rbrace=\vect{W}^T\matr{C}_{xx}\vect{W}$, the whitening condition becomes

\[
\vect{w}_i^T\matr{C}_{xx}\vect{w}_j = \delta_{ij} = \begin{cases}1, &i=j\\0 &i\neq
j\end{cases}
\]


We see that the usual orthogonality
condition $\vect{w}_i^T\vect{w}_j = \delta_{ij}$ of the standard GSO
should be changed so that the inner product is in whitening weighted by the covariance
matrix $\matr{C}_{xx}$ of the data.

We can get a whitening algorithm from the basic GSO by replacing all
the inner products by weighted inner products. This yields fhe
following algorithm:

\[
\begin{split}
\tilde{\vect{w}}_1 &= \vect{a}_1 \\
\tilde{\vect{w}}_j &= \vect{a}_j - \sum_{i=1}^{j-1}
(\vect{w}_i^T\matr{C}_{xx}\vect{a}_j)\vect{w}_i, \hspace{4mm} j=2,3,...,m \\
\vect{w}_j &= \tilde{\vect{w}}_j/\sqrt{\tilde{\vect{w}}_j^T\matr{C}_{xx}\tilde{\vect{w}}_j}
\end{split}
\]

Let us now check the validity of the whitening algorithm:

\[
\vect{w}_j^T\matr{C}_{xx}\vect{w}_j
=\frac{\tilde{\vect{w}}_j^T\matr{C}_{xx}\tilde{\vect{w}}_j}{\sqrt{\tilde{\vect{w}}_j^T\matr{C}_{xx}\tilde{\vect{w}}_j}\sqrt{\tilde{\vect{w}}_j^T\matr{C}_{xx}\tilde{\vect{w}}_j}}
= 1 \hspace{6mm} \text{OK}
\]
%
We have to show that
\begin{equation*}
\vect{w}_i^T  \matr{C}_{xx} \vect{w}_j  = 0 \mbox{\ \ \ \ $(i \neq j)$}
\end{equation*}
or equivalently $\tilde{\vect{w}_i}^T \matr{C}_{xx} \vect{w}_j  = 0$ ($i > j$).
   We proceed by induction:
\vspace{2mm}\\
1.\\
%\vspace{1mm}
\begin{align*}
\tilde{\vect{w}_2}^T \matr{C}_{xx} \vect{w}_1 &= 
\vect{a}_2^T \matr{C}_{xx} \vect{w}_1 - ( \vect{w}_1^T \matr{C}_{xx} \vect{a}_2 ) \times 
\vect{w}_1^T \matr{C}_{xx} \vect{w}_1 \nonumber \\
&= \vect{a}_2^T \matr{C}_{xx} \vect{w}_1
- \vect{w}_1^T \matr{C}_{xx} \vect{a}_2
 = 0 \mbox{.}
\end{align*}
%
We used $( \vect{a}_2^T \matr{C}_{xx} \vect{a}_1 )^T = \vect{a}_1^T \matr{C}_{xx} \vect{a}_2$, which holds
because both terms are scalars and $\matr{C}_{xx}^T = \matr{C}_{xx}$.
%x
\vspace{2mm}\\
2.\\
\vspace{2mm}
%
%
Suppose now that the condition
\begin{equation*}
\vect{w}_i^T \matr{C}_{xx} \vect{w}_k = \delta_{i,k} \mbox{\ \ holds for all $i,k < j$.}
\end{equation*}
%
For $i < j$,
\begin{align*}
\tilde{\vect{w}}_j^T \matr{C}_{xx} \vect{w}_i &= \vect{a}_j^T \matr{C}_{xx} \vect{w}_i - \sum_{k=1}^{j-1} 
( \vect{w}_k^T \matr{C}_{xx} \vect{a}_j ) ( \vect{w}_k^T \matr{C}_{xx} \vect{w}_i ) \nonumber \\
&=  \vect{a}_j^T \matr{C}_{xx} \vect{w}_i - \vect{w}_i^T \matr{C}_{xx} \vect{a}_j = 0 \mbox{,}
\end{align*}
because from the terms $\vect{w}_k^T \matr{C}_{xx} \vect{w}_i$ only the term
$\vect{w}_i^T \matr{C}_{xx} \vect{w}_i = 1$ remains and all the others are zero.
Thus it is proved that $\vect{w}_j^T \matr{C}_{xx} \vect{w}_i = 0$, $i < j$, as required.

b) Uniqueness: One can start from any set of linearly independent vectors
$\vect{a}_1, \vect{a}_2, ..., \vect{a}_n$ and get usually a different whitening
solution. Thus whitening is by no means unique. In fact, the whitening
condition

\[
\vect{W}^T\matr{C}_{xx}\vect{W}=\matr{I}
\]

would set in general $n^2$ conditions for the $n\times n$ dimensional whitening
matrix $\matr{W}^T$. But due to the symmetricity of $\matr{C}_{xx}$ there are
only $n(n+1)/2$ constraint conditions in performing whitening, giving
$n^2-n(n+1)/2=n(n-1)/2$ degrees of freedom.

\vspace{2mm}

\vspace{2cm}
\item a) Develop a whitening method based on principal component analysis.\\
b) Show that if the whitening matrix is multiplied by any orthogonal
matrix, the whitening property remains still valid.

---------------------------------------------------------------------------------------------

The PCA expression has as its basis vectors the eigenvectors
$\vect{e}_i$ of the data covariance matrix $\matr{C}_{xx}=E\lbrace
\vect{x}\vect{x}^T\rbrace$ assuming again that the vectors $\vect{x}$
have zero mean.

On the other hand, $\matr{C}_{xx}$ as any symmetric matrix can be expressed
in terms of its eigenvalue decomposition as

\[
\matr{C}_{xx} = \matr{E}\matr{D}\matr{E}^T = \sum_{i=1}^{n} \lambda_i \vect{e}_i \vect{e}_i^T
\]

where the matrix $\matr{E}=\left[ \vect{e}_1, \vect{e}_2, ...,
  \vect{e}_n \right]$ contains as its columns the
eigenvectors $\vect{e}_i$ of $\matr{C}_{xx}$ and the diagonal matrix $\matr{D}=\text{diag}(\lambda_1,
\lambda_2, ..., \lambda_n)$ the eigenvalues of $\matr{C}_{xx}$ in the same
order. Since the eigenvectors of a symmetric matrix are mutually
orthogonal

\[
\vect{e}_i^T\vect{e}_j = \begin{cases}1, &i=j\\0 &i\neq
j\end{cases}
\]


$\matr{E}^T\matr{E} = \matr{E}\matr{E}^T = \matr{I}$ where $\matr{I}$ is $n\times n$ unity matrix. Then we get

\[
\matr{E}^T\matr{C}_{xx}\matr{E} =
\matr{E}^T\matr{E}\matr{D}\matr{E}^T\matr{E} = \matr{D} \hspace{1cm} (*)
\]

The eigenvalues are positive so we can form a square root diagonal matrix

\[
\matr{D}^{-1/2} = \left[ 1/\sqrt{\lambda_1}, ..., 1/\sqrt{\lambda_n} \right]
\]

Multiplying $(*)$ by $\matr{D}^{-1/2}$ from the left and right yields

\[
\matr{D}^{-1/2}\matr{E}^T\matr{C}_{xx}\matr{E}\matr{D}^{-1/2}=\matr{I}
\]

Thus the matrix $\matr{D}^{-1/2}\matr{E}^T$ can be used for whitening
and the whitening transform by PCA is

\[
\textbf{y}=\vect{W}^T\vect{x} = \matr{D}^{-1/2}\matr{E}^T\vect{x}
\]

b) Whitening condition: $\vect{W}^T\matr{C}_{xx}\vect{W}=\matr{I}$

Multiply $\vect{W}$ by an orthogonal matrix $\matr{T}$

\[
(\vect{W}\matr{T})^T\matr{C}_{xx}\vect{W}\matr{T} = \matr{T}^T\vect{W}^T\matr{C}_{xx}\vect{W}\matr{T} = \matr{T}^T\matr{T} = \matr{I}
\]


\vspace{2mm}

\vspace{2cm}
\item Consider the problem of maximizing the variance of $y_m = {\bf
  w}_m^T{\bf x}\; (m=1, ..., n)$ under the constraint that ${\bf w}_m$
  must have unit Euclidian norm and orthogonal to all the
  previously-found principal vectors ${\bf w}_i$, $i<m$. Show that the solution
  is given by ${\bf w}_m={\bf e}_m$ the eigenvector of ${\bf C}_x$ corresponding to the mth
  largest eigenvalue.

---------------------------------------------------------------------------------------------

\begin{equation}
  \var\{ y_m \} = \E \{ y_m^2 \} = \E \{ \vw_m^T \vx \vx^T \vw_m^T \}
  = \vw_m^T \E \{ \vx \vx^T \} \vw_m^T = \vw_m^T \mC_x \vw_m
\end{equation}

The constraints:
\begin{equation}
  \vw_k^T \vw_l = \begin{cases}1 \quad\text{if $k=l$} \\
    0 \quad\text{if $k \neq l$} \end{cases}
\end{equation}

The Lagrangian function:
\begin{equation}
  {\cal L}(\vw_m, \vects{\lambda}) = \frac{1}{2} \vw_m^T  \mathbf{C}_x \vw_m
    + \lambda_0 ( \vw_m^T \vw_m - 1) + \sum_{i=1}^{m-1} \lambda_i \vw_m^T \vw_i
\end{equation}
and its derivative
\begin{equation}
  \label{eq:prob_6_1_a}
  \nabla {\cal L} = {\bf{C}}_x \vw_m + \lambda_0 \vw_m
     + \sum_{i=1}^{m-1} \lambda_i \vw_i.
\end{equation}

Letting $m=1$ and setting the derivative to zero yields
\begin{equation}
  \mC_x \vw_1 = - \lambda_0 \vw_1
\end{equation}
which implies that $\vw_1$ is an eigenvector of $\mC_x$.

Evaluating the variance of $y$:
\begin{equation}
  \var\{ y_1 \} = \var\{ \vw_1^T \vx \} = \E\{ \vw_1^T \vx \vx^T \vw_1 \}
    = \vw_1^T \mC_x \vw_1 = \vw_1^T (- \lambda_0 \vw_1) = - \lambda_0.
\end{equation}
This is maximized when $- \lambda_0$ is equal to the largest
eigenvalue of $\mC_x$ and $\vw_1$ is the corresponding eigenvector.

Let us now continue inductively.  For $m>1$, the previous weight
vectors $\vw_i$, $i = 1, \dots, m-1$ are the eigenvectors
corresponding to $m-1$ largest eigenvalues.  By multiplying
Eq.~(\ref{eq:prob_6_1_a}) from the left by $\vw_i$, $i = 1, \dots,
m-1$, we get $\lambda_i = 0$ for all $i$.  Thus again
\begin{equation}
  \mC_x \vw_m = - \lambda_0 \vw_m.
\end{equation}
Thus $\vw_m$ is again an eigenvector and to maximize the variance it
must be the one with the largest remaining eigenvalue.

NB: Remember that $\mC_x$ is a symmetric matrix and thus its
eigenvalues are real and eigenvectors orthogonal.




\vspace{2mm}

\vspace{2cm}
\item The learning rule for a PCA neuron is based on maximization of
  $y=({\bf w}^T{\bf x})^2$ under constraint $ \| {\bf w} \| = 1$. (We have now omitted the
  subscript 1 because only one neuron is involved.)

Show that an unlimited gradient ascent method would compute the new
vector ${\bf w}$ from
\[
{\bf w} \leftarrow {\bf w} + \gamma ({\bf w}^T{\bf x}){\bf x}
\]
with $\gamma$ the learning rate. Show that the norm of the weight
vector always grows in this case.

---------------------------------------------------------------------------------------------


\begin{equation}
  \frac{\partial y}{\partial \vw} = 2 \vx \vx^T \vw = 2 (\vw^T \vx) \vx.
\end{equation}
Thus the gradient ascent updates are
\begin{equation}
  \vw \leftarrow \vw + \gamma (\vw^T \vx) \vx.
\end{equation}

Evaluating the norm of the new iterate
\begin{equation}
  \begin{split}
    ||\vw + \gamma (\vw^T \vx) \vx||^2
      &= \left(\vw + \gamma (\vw^T \vx) \vx\right)^T
         \left(\vw + \gamma (\vw^T \vx) \vx\right) \\
      &= \vw^T \vw + 2 \gamma \vw^T (\vw^T \vx) \vx + \gamma^2 (\vw^T \vx)^2 \vx^T \vx \\
      &= ||\vw||^2 + 2 \gamma (\vw^T \vx)^2 + \gamma^2 (\vw^T \vx)^2 ||\vx||^2
       \ge ||\vw||^2
  \end{split}
\end{equation}



\end{enumerate}
\end{document}             % End of document.
