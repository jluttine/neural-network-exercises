\documentclass[10pt]{article}
\usepackage[english]{babel}
\usepackage[latin1]{inputenc}
\usepackage{subfigure}
\usepackage{epsfig}
\usepackage{amsmath,amssymb}
\parindent 0mm
\textwidth 16cm
\textheight 23cm
\oddsidemargin 0cm
\evensidemargin 0cm
\topmargin -10mm
\newcommand{\vect}[1]{{\bf{#1}}}
\newcommand{\svect}[1]{\boldsymbol{#1}}
\newcommand{\matr}[1]{\boldsymbol{#1}}

\renewcommand{\vec}[1]{\mathbf{#1}}
\newcommand{\set}[1]{\mathcal{#1}}
\newcommand{\C}{\set{C}}
\newcommand{\E}{\mathcal{E}}
\newcommand{\I}{\vec{I}}
\renewcommand{\L}{\mathcal{L}}
\newcommand{\N}{\mathrm{I \negmedspace N}}
\newcommand{\R}{\mathrm{I \negmedspace R}}
\newcommand{\V}{\set{V}}
\newcommand{\W}{\vec{W}}
\newcommand{\X}{\set{X}}
\newcommand{\e}{\vec{e}}
%\newcommand{\f}[1]{\mathrm{#1}} %funktio
\newcommand{\h}{\vec{h}}
\newcommand{\m}{\vec{m}}
\newcommand{\mub}{\boldsymbol{\mu}}
\newcommand{\n}{\vec{n}}
\renewcommand{\t}{\vec{t}}
\renewcommand{\u}{\vec{u}}
\renewcommand{\v}{\vec{v}}
\newcommand{\w}{\vec{w}}
\newcommand{\x}{\vec{x}}
\newcommand{\y}{\vec{y}}
\newcommand{\Y}{\vec{Y}}
\newcommand{\z}{\vec{z}}
\newcommand{\argmin}{\operatornamewithlimits{argmin}}
\newcommand{\argmax}{\operatornamewithlimits{argmax}}


\begin{document}
\pagestyle{empty}
\begin{Large}
\begin{bf} 
T-61.5130 Machine Learning and Neural Networks\\ 
\end{bf}
\end{Large}
Karhunen, Hao Tele\\  
\\
\begin{large}
\begin{bf}
Exercise 8,  1.12.2011
\end{bf}
\end{large}
\begin{enumerate}

\item Starting with the primal problem for the optimization of the
  separating SVM-hyperplane for nonseparable patterns, formulate the
  dual problem for the nonseparable case.

\vspace{2mm}

\item In the case of separable classes, what happens to the separating
hyperplane of SVMs if the amount of noise increases? How about the
case of nonseparable classes?

\vspace{2mm}

\item The inner-product kernel for a polynomial learning machine is
  defined by
  \begin{displaymath}
    K(\x,\x_i) = (1 + \x^T \x_i)^p \; ,
  \end{displaymath}
  where $p$ is a positive integer.  What is the dimension of the
  implicit feature space if the dimension of $\x$ is $m$.  Find lower
  and upper bounds if you cannot find the exact answer.

\vspace{2mm}

\item Can a support vector machine be used to solve a pattern
  classification task where there are more than two classes?


\end{enumerate}
\end{document}             % End of document.
